\chapter{Introduction}

\section{Overview of Distributed Sensor Networks}
Distributed sensor networks consist of any number of sensors that collect information about the environment around them. The sensors that make up these networks can either be homogeneous or heterogeneous. Distributed sensor networks are dynamic in that sensors can be added or removed from the network at any time. With the onset of the Internet of Things (IoT), it’s easier than ever to build and deploy distributed sensor networks. Further, mobile devices, such as mobile phones, are seeing increased usage as intelligent sensing agents.

With all of these factors combined, management, collection, detection, localization, and analysis from distributed sensor networks is not a trivial task. The data collected from distributed sensor networks under certain circumstances can be considered Big Data. That is, distributed sensor networks create a large volume of data due to the abundance of IoT and mobile devices. Distributed sensor networks create a variety of data with different formats and data quality issues. Distributed sensor networks can produce data at high velocity. These characteristics of data produced from distributed sensor networks create a need for efficient architectures and specific algorithms designed for working with Big Data.

Further, sensor networks are often constrained in both computing power and available energy sources. This forces us to find comprises between data collection, onboard sensor processing, sensor communication, and network coordination.

This proposal focuses on identifying and tuning several features of distributed sensor networks. 

Triggering is the act of monitoring low fidelity feature extracted data and triggering sensors for high fidelity data streams when abnormalities are seen in the triggering data stream.

Detection is the algorithmic ability to detect signals of interest among a sea of background noise. Sensor networks often detect signals without knowing what they are or how to classify them. 

Classification is the act of associating a signal with a known designation. 

\subsection{Distributed Sensor Networks and Big Data}

\section{The Data Management and Analysis Problems}

\section{Traditional Approaches to Data Management and Analysis}

\section{Seven proposed benefits of Laha} \label{laha-benefits}
\subsection{Tiered management of Big Data}
\subsection{Automatically provide context to classified incidents}
\subsection{Adaptive optimizations for detection and classification}
\subsection{Adaptive optimizations for triggering}
\subsection{Provides a model of underlying sensor field topology}
\subsection{Decreased bandwidth of entire DDS}
\subsection{Decreased energy usage from optimized communications and analysis}

\section{Evaluation of Laha}
\subsection{Design and implement Laha-compliant software reference implementations (OPQMauka and Lokahi)}
OPQMauka is a distributed, plugin-based middleware component of the Open Power Quality (OPQ) software stack that provides higher level analytics and data management for a distributed PQ network.

\subsection{Deploy Laha reference implementations on test sites}
\subsection{Validate data collected by Laha deployment}
\subsection{Use Laha deployment to evaluate each of the seven proposed benefits discussed in section \ref{laha-benefits}}

\section{Anticipated contributions of this thesis}
\subsection{Laha design: a novel distributed sensor network adaptive design with seven useful properties}
\subsection{Laha evaluation: empirical data to confirm or deny the seven useful properties}
\subsection{OPQMauka and Lokahi: reference implementations of Laha}
\subsection{Implications for modern distributed sensor networks}

\section{Timeline}
\subsection{2018 Q4: Implement, deploy, and validate Laha reference implementations}
\subsection{2019 Q1: Begin validated data collection}
\subsection{2019 Q2: Continue validated data collection, thesis chapters 1-3}
\subsection{2019 Q3: Finish validated data collection, these chapters 4-6}





