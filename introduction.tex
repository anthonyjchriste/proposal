\chapter{Introduction}

\section{Overview of Distributed Sensor Networks}
Distributed sensor networks (DSNs) consist of any number of sensors that collect and sense information about the physical environment around them. The sensors that make up these networks can either be homogeneous or heterogeneous. Distributed sensor networks are dynamic in that sensors can be added or removed from the network at any time. DSNs are also increasingly including mobile sensors as well. With the onset of the Internet of Things (IoT), it’s easier than ever to build and deploy distributed sensor networks. Further, mobile devices, such as mobile phones, are seeing increased usage as intelligent sensing agents.

\subsection{DSN Data Collection Schemes}
Data can be collected from DSNs in various ways. Sensors can store data onboard to be collected manually or transmit data via a multitude of mediums (satellites, radio, laser, wired connections) using a slew of standards (TCP/IP, Zigbee, Bluetooth, custom, etc).

In some DSNs, data is routed between the sensors using various approaches % TODO cite this
and eventually makes its way to a sink or sinks. A sink is a data collection node. Cloud computing has become a prevalent choice for DSNs sinks since they often provide the ability to provision resources as needed. Various approached have been proposed that minimize and optimize communications between sensors within a DSN. % TODO Cite this

In other DSNs, sensor nodes have direct access to a sink, and instead of communicating with each other and passing messages among themselves, the nodes send their data directly to the sink. It's also possible that a DSN takes a hybrid approach and performs some communication within the network and some directly to the sink.

Not only are there various approaches to routing data, but there are also various approaches to deciding what kind of data to send (or acquire).

\subsection{DSN Detection Schemes}
On one extreme end we have the send everything approach where each sensor sends its data to the sink all the time (or at least when it has the means to do). In this scenario, the sink is responsible for collection, cleaning, detection, and classification of signals and events within the data. This approach can be bandwidth and energy intensive but provides the benefit of allowing more complete analysis to occur beyond the sink where more computational resources exist. This provides more accurate results and allows the sink to examine the results in aggregate with the rest of the DSN.

The other extreme is that sensors only send data when they have detected a signal of interest. In this scenario, sensors use onboard computing capabilities to filter their sensor stream and perform signal detection on the device. Only when the devices make a detection do they send the detection or data stream to a sink for further processing. This minimizes bandwidth but detection and classification of signals must occur with more constrained computing and energy environments. Further, the global state of the network can't be known without adding the complexity of sensor-to-sensor communication.

Often time a hybrid approach is taken where low fidelity feature extracted and sometimes aggregate data is sent from the sensors to the sink. The sink analyzes the low fidelity feature extracted stream to determine if raw data should be requested from the sensors. The act of requesting data from the sensors is called triggering. This approach is useful because we still gain bandwidth benefits and can easily gain an understanding of the global state of a network.

With all of these factors combined, management, collection, detection, localization, and analysis from distributed sensor networks is not a trivial task. DSNs can produce massive amounts of data. The emergence of IoT has increased the heterogeneity of sensors with multiple hardware configurations, variations of data APIs, and incomplete or bad sensor data. For these reasons, the data collected from distributed sensor networks under certain circumstances is considered Big Data.

\subsection{DSN Classification Schemes}
% TODO

\subsection{DSNs as Big Data}
Big Data is generally defined by the four V's; volume, velocity, variety, and value. These characteristics can be observed in many of the DSNs that exist and are being created today. % TODO more citations

That is, distributed sensor networks create a large volume of data due to the abundance of IoT and mobile devices that make up DSNs. As communication infrastructures improve and hardware becomes smaller, smarter and more energy efficient, sensors are able to send and transfer larger amounts of data. The ease of building and deploying sensors in DSNs means that more sensors can be produced much more cheaply allowing for more sensors to be used within a DSN, increasing coverage, but also increasing the volume of data.

Distributed sensor networks create a variety of data with different formats and data quality issues. Distributed sensor networks can produce data at high velocity. These characteristics of data produced from distributed sensor networks create a need for efficient architectures and specific algorithms designed for working with Big Data.

Further, sensor networks are often constrained in both computing power and available energy sources. This forces us to find comprises between data collection, onboard sensor processing, sensor communication, and network coordination.

This proposal focuses on identifying and tuning several aspects of distributed sensor networks.

Triggering is the act of monitoring low fidelity feature extracted data and triggering sensors for high fidelity data streams when abnormalities are seen in the triggering data stream.

Detection is the algorithmic ability to detect signals of interest among a sea of background noise. Sensor networks often detect signals without knowing what they are or how to classify them.

Classification is the act of associating a signal with a known designation.

\section{The Data Management and Analysis Problems}

\section{Traditional Approaches to Data Management and Analysis}

\section{Seven proposed benefits of Laha} \label{laha-benefits}
Laha is designed to adaptively optimize the collection, triggering, detection, and classification of signals within a DSN. These optimizations provide several benefits to DSNs.

\subsection{Tiered management of Big Data}
Laha provides tiered management of the Big Data that the framework consumes. This is mainly accomplished using the layered approach that Laha provides. All data within the Laha framework is garbage collected using a configurable time to live (TTL) for each layer. As data moves from the bottom to the top of the framework, noise is discarded and only interesting data as determined by the higher layers is preserved and forwarded upwards within the framework. In this way, the network can be tuned to preserve increasingly important data. The details of Laha's tiered approach can be found in section \ref{big-data-management}.

\subsection{Automatically provide context to classified incidents}
Laha provides Annotations Phenomena (see \ref{annotations-phenomena}) that allows users or algorithms to tag signals of interest with contextual information. There is already a large amount of research for providing classifications of signals of interest, however annotations provide context about the classifications themselves. That is, Laha provides the ability to assign causality to already classified signals.

Initially, a library of annotations is required to be built for a particular DSN. Once the library has been built, Laha can provide automatic annotation assignment using similarity metrics. By using annotations and determining causality, it's possible to create actionable responses to signals observed within a network.

Further, annotations can be used to optimize detection and classification of known signals. For example, imagine a power quality network that observes a voltage sag on the same sensors at the same time periodically. If the cause of the voltage sag can be determined (such as a motor turning on), then detection of this signal can either be muted or analyzed more deeply. Also, other voltage sags that exhibit similar characteristics can then be automatically annotated.

\subsection{Adaptive optimizations for triggering}
Many triggering schemes rely on thresholds being surpassed within feature extracted data streams. For example, triggered data streams in a power quality network might consist of voltage, frequency, and THD extracted features to determine if there is likely a signal of interest observed from a given set of sensors. If any of the extracted features surpass a preset threshold, then we  end up triggering the devices for raw, higher fidelity data. However, there are cases where the feature extracted stream may not surpass a predefined threshold and those sensors would not be triggered.

By using different types of Phenomena, we can improve our triggering efficiency. For example, Locality Phenomena, as discussed in section \ref{locality-phenomena}, allow us to predict detections and classifications in space. If a particular grouping of sensors that are related in space always observe the same detections and classifications, then we can build a predictive model of when the framework can expect to see those things. In this way, a sensor may trigger on a passed threshold, but other sensors that are co-located may not trigger because feature extracted data is below the triggering threshold. If the Locality Phenomena predicts that other co-located sensors should have detected the same signal, then we might trigger those devices for high fidelity data to determine if the signal is in the raw data stream even though it didn't meet the triggering threshold.

Laha can use Periodicity Phenomena, as discussed in section \ref{periodicity-phenomena}, in similar ways. When periodic signals are classified, we can create Future Phenomena that predicts when signals of interest should be detected and classified. This allows Laha to optimize triggering by tuning the triggers to specifically look for Periodic or Future Phenomena. Periodic and Future Phenomena also allows Laha to tune classification algorithms to the predicted classifications.

\subsection{Adaptive optimizations for detection and classification}
Not only can Laha optimize triggering, but similar usages of Locality and Future phenomena can be utilized to improve detection and classification efficiency. With Locality Phenomena, a model of common detections and classifications can be built for a set of co-located sensors. Detection and classification algorithms can be tuned to search for specific signals that are often observed within this model, pruning the search space and increasing the accuracy of detection and classification algorithms.

Future and Periodic Phenomena provide the same benefits to detection and classification as Locality Phenomena. That is, if Laha is able to predict when a signal is going to arrive and also predict how that signal is going to be classified, then it can tune its detection and classification algorithms specifically for the signal of interest.

\subsection{Provides a model of underlying sensor field topology}
Predictive phenomena use localization of signals to build communities or groupings of sensors that observe similar signals in both time and space. Over time, Laha can begin to build a model of the underlying topology of the sensing field. For instance, if a group of sensors always see the same signal, then we can assume that either the signal has a far reach, or the sensors are grouped together geographically. With enough sensor penetration, we can differentiate between these two scenarios.

Further, if the sensors provide any sort of location information, Laha can provide a mapping from physical location to sensor field topology. This is especially useful when the topology of the sensor field is not known a priori. For example, smart meters under Laha can build a model of an electrical grid which is generally only information that utilities can provide.


\subsection{Decreased bandwidth of entire DSN}
One of the tertiary benefits of optimizing triggering and classification is that we get decreased bandwidth usage for free. By improving triggering efficiency, its possible to trigger less devices for raw data when something interesting is observed in the triggering stream.

Predictive phenomena allow us to target triggers to sensors that are most likely to have captured a signal of interest. For instance, if a geographical grouping of sensors tend to always observe the same signals of interest, then we can use that grouping to optimize triggering. Instead of triggering all devices in an area to search for the signal, we can trigger only on the specific grouping that is most likely have observed the signal. We can also selectively ignore sensors within a predictive grouping if Laha assumes that all sensors observed the signal, then we may only need to analyze the signal from one of the sensors rather than all of the sensors in the grouping.

\subsection{Decreased energy usage from optimized communications and analysis}
In general, sensors are both compute and energy constrained. Most of the time, communications are the main source of energy usage within a DSN.

In an energy constrained DSN where communications amount for the most usage of energy, decreased bandwidth also provides decreased energy usage. This can prolong the life of DSN or allow energy usage for other parts of the network (such a pushing computations to the edge).

\section{Evaluation of Laha}
\subsection{Design and implement Laha-compliant software reference implementations (OPQMauka and Lokahi)}
OPQMauka is a distributed, plugin-based middleware component of the Open Power Quality (OPQ) software stack that provides higher level analytics and data management for a distributed PQ network.

\subsection{Deploy Laha reference implementations on test sites}
\subsection{Validate data collected by Laha deployment}
\subsection{Use Laha deployment to evaluate each of the seven proposed benefits discussed in section \ref{laha-benefits}}

\section{Anticipated contributions of this thesis}
\subsection{Laha design: a novel distributed sensor network adaptive design with seven useful properties}
\subsection{Laha evaluation: empirical data to confirm or deny the seven useful properties}
\subsection{OPQMauka and Lokahi: reference implementations of Laha}
\subsection{Implications for modern distributed sensor networks}

\section{Timeline}
\subsection{2018 Q4: Implement, deploy, and validate Laha reference implementations}
\subsection{2019 Q1: Begin validated data collection}
\subsection{2019 Q2: Continue validated data collection, thesis chapters 1-3}
\subsection{2019 Q3: Finish validated data collection, thesis chapters 4-6}





