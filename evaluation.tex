\chapter{Evaluation}
Evaluation of the Laha framework involves deploying reference Laha-compliant DSNs, validating the data collected from the reference implementations, and then comparing and contrasting various metrics for each of the proposed benefits. Metrics will be collected during a set of experiments for each of the Laha reference implementations in early 2019. 

\section{Deploy Laha reference implementations on test sites}
In Q4 2018, 10 to 20 Laha-compliant OPQBoxes will be deployed on the University of Hawaii at Manoa's power microgrid. Using a provided blueprint of the microgrid as a guide and collaborating with the Office of Energy Management, these sensors will be placed strategically with the hopes of observing PQ signals on the same line, PQ signals generated from intermittent renewables, local PQ signals, global PQ signals, and PQ signals near sensitive lab electronics. Many of these sensors will be co-located with industry standard PQ monitoring systems. The industry standard sensors provide both ground truth and a means of comparison between a Laha designed network and a non-Laha designed network.. 

In Q4 2018,  20 to 30 Laha-compliant Lokahi sensors will be deployed near and around the Infrasound Laboratory in Kailua-Kona on the Big Island of Hawaii. These sensors will be placed strategically around a calibrated infrasound source. The sensors will be placed with the assistance of Dr. Milton Garces to ensure that I can target sensors at different distances by tuning the amplitude and frequencies of the infrasound signal. In this way, I know which devices should or should not have received the signal.

\section{Validate data collected by Laha deployment}
Beginning in Q1 2019, I will begin validated data collection from both the OPQ network and the Lokahi network. 

Data will be validated in the OPQ network be comparing detected and classified signals against industry standard meters that are co-located with our sensors. Data validation will be an autonomous process that validates signals and trends seen in both the industry sensor and the OPQ sensors. Data validation will provide metrics for signals and trends that the reference sensors observed but OPQ sensors did not (false negatives) as well as signals that the OPQ sensors observed and the reference sensors did not (false negatives).  Specifically, I will be looking to compare long term trends (voltage, frequency, and THD readings over a time period of days) as well as more transient signals of interest (i.e. voltage sags/swells, frequency variations, excessive THD, and outages).

Data from the Lokahi network will be validated against industry standard infrasound sensors. We also control the amplitude and frequency of the signals generated from the calibrated infrasound source and can use geophysical equations to predict which sensors should have seen or not seen an infrasonic signal. Data validation is autonomous for this network as well. Similar to the OPQ network, I will be collecting metrics on false positive and false negatives as compared to the reference sensors. 

Data validation for both networks will continue for all data collection until the end of the project.

\section{Use Laha deployments to evaluate each of the five proposed benefits discussed in section \ref{laha-benefits}}
The Laha deployments for both OPQ and Lokahi will be used to evaluate each of the proposed benefits discussed in section \ref{laha-benefits}. Each deployment offers different techniques for performing evaluation. 

In the OPQ deployment, OPQBoxes are deployed and co-located with industry standard, calibrated, reference sensors. Each of these sensors cost thousands to obtain and install, collect all the data all the time, and can only be connected to the power main as it enters a building. [TODO discuss the actual hardware and characteristics of the sensors installed by the office of power management at uh here]. These sensors provide a means for verifying signals received or not received by OPQ, as well as confirming long term trend data. I have been provided access to these sensors and stored data via the Office of Energy Management at UH Manoa. The data is accessible via an HTTP API. The Office of Energy Management at UH Manoa has also provided the full schematics for the UH power grid. This will be used as a ground truth for topology estimates and distributed signal analysis. OPQBoxes are placed in strategic locations on the UH Manoa campus specifically in order to evaluate the distributed nature of PQ signals. For example, OPQBoxes are placed on the same electrical lines as well as separate electrical lines to observe how PQ signals travel through an electrical grid.
% TODO discuss the actual hardware and characteristics of the sensors installed by the office of power management at uh here

In the Lokahi deployment, I have the opportunity to generate infrasound signals using a calibrated infrasound source \cite{park2009rotary}.. The source can be tuned to produce infrasound at configurable frequencies and amplitudes. The source works by attaching a variable pitch propeller to an electric motor that can be driven by a waveform generator. The source can generate signals that can be observed at large stand off distances, over tens of kilometers. Similar to the OPQ deployment, sensors within the Lokahi deployment will be co-located with industry standard, calibrated, infrasound sensors. These sensors can provide a metric of signals that were correctly observed, incorrectly observed, or not observed at all by the Lokahi deployment. [TODO characterize and provide details on these sensors] Further, infrasound itself is characterized quite well by various geophysical equations. These equations can be used to predict if sensors deployed in the Lokahi deployment are likely to observe generated infrasound signals.
% TODO discuss the actual hardware and characteristics of the sensors installed by the office of power management at uh here

\subsection{Evaluation of Tiered Management of Big Data}\label{eval-big-data}
The goal of tiered management of Big Data is to add a mechanism that provides a maximum bounds on storage requirements of sensor data at each level in the Laha hierarchy while simultaneously reducing sensor noise as Laha Actors move ``interesting" data upwards. This in turn should decrease the amount of false positives since forwarded data is more likely to include signals of interest and less likely to be sensor noise. 

It's not yet to be seen if I will see a decrease in false negatives. On the one hand, it's possible that Laha will throw away data that did contain signals of interest. In this case, detection or classification Actors will not observe the signals because the data has been discarded leading to increased false negatives. On the other hand, by reducing false positives and increasing the signal-to-noise ratio as data moves upward, Phenomena has a better chance of optimizing triggering, detection, and classification which may in turn inform Laha to save data that would have been previously thrown away. In this way, it's possible that Laha will reduce false negatives.

We will evaluate the number of false positives and false negatives in detections, classifications, and Phenomena compared against industry standard reference sensors. A positive outcome for this metric would be a reduction in both false positives and false negatives compared to an approach that does not use tiered data management. A negative result would be an increase in either false positives or false negatives. 

During the acquisition and curating of data, metrics will be collected and stored about how much data is saved (in bytes) versus how much data is discarded at each level within the Laha data hierarchy. These numbers will be compared against data storage as if the OPQ and Lokahi frameworks were to take a ``store everything" approach. Evaluation metrics provided will include percentage of data storage saved per data hierarchy level as well as an estimate of overall decrease in data storage requirements for the entire DSN. A positive result from these metrics would show significant reduction in storage requirements for each level in the framework compared against a ``store everything approach" and other state-of-the-art data storage solutions.

I will also provide metrics on ``continuous storage pressure" which is a measure of the average amount of data storage required at each level given the current state of the network. That is, since data at all lower levels of the framework assigns a TTL to the data within the collection, the collection will exhibit a constant data pressure during sensor data collection. For example, at the lowest level, the IML collects raw data from all sensors all the time. Given the sample rate per sensor, the size per sample, the number of sensors, and a known TTL for this level, I can estimate the maximum bounds of data management requirements that the IML requires. We can play similar estimation games with higher levels of the framework. I will compute the statistical error between the predicted storage pressure and the actual storage pressure recorded during the experiments. A positive outcome would show strong correlation between the predicted storage pressure and the actual storage pressure. A negative outcome would show weak correlation between the predicted and actual values.

Finally, I will provide an evaluation that weighs the results of all three metrics against each other. For example, if I see positive results for data storage reduction and negative results for false positives, do the benefits of the data storage reduction outweigh the negatives of increased false positives?

I would expect that DSNs that have a lower signal-to-noise ratio will see greater benefits from tiered data management than DSNs that already have a decent signal-to-noise ratio.

\subsection{Evaluation of Contextualizing Classified Incidents}
Adding context to classified Incidents is the act of providing a statistical likelihood of the underlying cause of the Incident. These include things like showing that a voltage sag is caused by turning on the dryer every day at 2PM or an identifying as infrasound signal as a repeitive flight pattern near an airport. Context is provided by external sources to the DSN (such as users or by performing data fusion with other correlating data sets).

Evaluating contextualized events consists of setting up experiments where I assign context for a specific set of signals and resulting Incidents. Then testing to see if Phenomena are able to correctly apply context to Incidents when the same signals are generated again. I will record the number of false positives and false negatives for assigning context to Incidents.

A postive result would be to see the correct context applied to incidents more than half of the time. That is, I expect context to be applied correctly to at more than 50\% of Incidents for which context has been previously defined.

I expect to see contextualization work better in DSNs where signals provide more meaures for discrimination. For example, PQ networks contain many different types of classified PQ signals, however thereis a small subset of causes atributed to each type of PQ signal classification.This decreases Laha's search space and in theory should make it easier to provide context.

[TODO: This is still probably my weakest evaluation and I need to define it better still.]

\subsection{Evaluation of Adaptive Optimizations for Triggering}
Triggering is the act of observing a feature extracted data stream for interesting features and triggering sensors to provide raw data for a requested time window for higher level analysis. Adaptively optimizing triggering is a way to tune triggering algorithms and parameters with the aim of decreasing false positives and false negatives. In this context, a false positive is triggering on a data stream that does not contain a signal of interest and a false negative is not triggering on a data stream that does contain a signal of interest. 

Adaptive triggering is only useful in networks that utilize triggering. Specifically, this technique can not be applied to DSNs that take a collect everything all the time approach.

Even if a DSN utilizes triggering, it's not clear that adaptive triggering even takes place. The first question I will evaluate is, does adaptive optimization of triggering take place at all given the domain of the DSN? That is, does the nature of the underlying sensor field contribute to optimization of triggering? I will compare if and how optimizations take place in the two reference networks for the domains of PQ and infrasound.

In order to evaluate triggering efficiency within our Laha deployments, Laha will only adaptively modify triggering for half of the devices in the OPQ deployment. In the Lokahi deployment, I will run the same experiment twice. The first run will not optimize triggering and the second run will optimize triggering.

Once the experiments have been run, I will first determine if optimization of triggering has occurred, and if it did, compare the number of false negatives and false positives against the runs that did not use optimized triggering or where optimization did not occur. 

Triggering can also have significant impacts on overall sensor power requirements and DSN bandwidth requirements. Many of the optimizing triggering algorithms present in the literature exist to minimize sensor energy requirements and bandwidth requirements. This is addressed in great detail in the literature review by Anastasi et al. \cite{anastasi_energy_2009}. This is accomplished by reducing communications between sensor nodes and the sink. It's argued in \cite{pottie2000wireless} that the cost of transmitting a single bit of information from a sensor cost approximately the same as running 1000 operations on that sensor now. However, there is some contention on this topic as \cite{alippi_adaptive_2010} argues that in some modern sensors computational requirements can equal or eclipse those of  sensor communication.  

I hope to show that a side effect of Laha's optimized triggering is reduced bandwidth and sensor energy requirements. To this end, I will calculate metrics for total data sent and received at the sink node of each network for each device in the network. A positive result would show decreased bandwidth usage for devices that utilize optimized triggering. A negative result would show similar or more bandwidth usage for devices that utilize optimized triggering.

I further hope to show that another benefit of Laha's optimized triggering is reduced sensor energy requirements. The evaluation for this metric will occur with the Lokahi network where sensors can be dependent on batteries. I will run two experiments. For each experiment, all sensors will be charged to battery level of 100\%. In the first experiment, I will not utilize optimized triggering. In the second experiment I will utilize optimized triggering. In both experiments, I will measure the final battery level after the experiment and also measure how quickly the battery depletes for each sensor. This is possible because data in the Lokahi network contains timestamped entries with battery levels.

\subsection{Evaluation of Adaptive Optimizations for Detection and Classifications}
Detections occur when triggering observes something ``interesting" in the feature extracted data stream. A Detection is a contiguous window of raw sensor data that was requested by triggering that may or may not contain signals of interest. Optimizing detections involves optimized the window sizes to increase the signal-to-noise ratio of the window. Fine grained features are then computed by Detection Actors and moved to the Incidents Level where classification of signals takes place. Optimizing Detections involves trimming detection windows to increase signal-to-noise. Optimizing of classifications for Incidents involves tuning parameter sets for the underlying classification algorithms. 

Predictive and Locality Phenomena as well as toplogy optimizations will be used to provide optimizations to the Detections and Incidents levels. 

Evaluation of adaptive optimizations for detection and classification within the Laha network will be conducted differently for each Laha deployment.

In the Lokahi deployment, I will control the production of infrasound signals using the available infrasound source. I will run two experiments, where the amplitudes and frequencies of the signals are the same and the locations of the devices remain invariant. In the first experiment, Laha will not use optimized detection or classification provided by Phenomena. In the second experiment, Laha will use optimized detection and classification techniques provided by Phenomena. 

With known frequencies and amplitudes of the infrasound signals, I can compare the rate of detections and classifications between the optimized and unoptimized experimental runs. I expect to see a greater number of and more accurate detections and classifications from the optimized experiment.

In the OPQ deployment, I will compare the same metrics as the Lokahi deployment, but instead of controlling the source signal, I will co-locate OPQBoxes. In each pair of co-located OPQBoxes, one will be analyzed using Phenomena optimized detection and classification algorithms and the other will be analyzed using unoptimized detection and classification algorithms.

I will collect and evaluate the number of false positves and false negatives for Incidents generated with optimization and without optimization. A positive outcome would include a decrease in either false positives, false negatives, or both. A negative result would be an increase in either or both false posistives or false negatives.

I will also calculate the signal-to-noise ration in Detections to determine if optimization of detections is working. A posistive outcome would be an increase in the signal-to-noise ration and a negative outcome would be similar or a decrease in signal-to-noise ratio.

\subsection{Evaluation of Model of Underlying Sensor Field Topology}
Laha hopes to build a model of the underlying sensing field topology. This is not the topology of the physical layout of the sensors (this is generally already known apriori or by collecting location information), but rather the toplogy by which signals travel. For example, in a PQ network the topology is the physical power grid and switches that PQ signals travel through. In an infrasound network, the toplogy is the atmosphere through which sound waves travel. Laha aims to build a statistical model of the distances between sensors according to the toplogy of the sensing field by observing recurrent incidents over time. This can perhaps shed some light on understanding the topology of a sensing field without knowing anything about it before hand.

To evaluate the model of the sensing field topology, I will take two different approaches for each Laha deployment. In both deployment, the sensing field toplogy is known beforehand to provide a ground truth. I will then compare Laha's computed signal distance between sensors to the actual signal distance between sensors as provided by the ground truths.

In the Lokahi deployment, sensors will be strategically placed at different distances from an infrasound source. Some sensors will be close to each other geographically, but separated by terrain that infrasound signals will not easily travel through. By moving the infrasound source, I can expect to see infrasound signals arriving or not arriving at the sensors depending on the source and direction of the signal along with the physical features of the land. By performing multiple experiments, I hope to provide a model of the physical environment topology that Laha has built. I will compare Laha's model to the known topology and provide a statistical error analysis. 

In the OPQ deployment, sensors will be strategically placed on like and unlike electrical lines to observe how distributed PQ signals move through a power grid. In this deployment, Laha will build a topology model that doesn't show physical geographic distance between sensors, but instead will build a model of the electrical distance between sensors. This data will be evaluated by comparing the electrical distances found by the Laha model to the actual UH power grid as referenced by the schematic provided by the Office of Energy Management at UH Manoa. A statistical error analysis of the differences between electrical distances between the model and the schematic will be provided as an evaluation metric.

A positive outcome would be to show that there is high correlation between the Laha signal distances and the ground truth distances. A negative outcome would show low correlation.

Assuming high correlation and a statistical model of the sensiing field, I would like to evaluate if Laha is able to use this information to optimize triggering, classification, or predictive analytics. In order to evaluate this, I will collect the number of false positives and false negatives at all levels in the Laha hierarchy while optimizing from topolgy and without optimizing from topolgy. I expect to see less false positives and less false negatices when utilizing toplogy optimizations. A negative result would be a larger number of false positives or false negatives.

I expect to only see results in networks where signals travel fast enough to create a stistical differnce between arrival times at the various sensors. In sensing fields where signals travel slowly and uniformally (i.e. a temperature collection DSN), it may be more difficult or impossible to actually determine the sensing field topology.
