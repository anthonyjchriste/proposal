\chapter{Design of Laha}

Laha, which means, to spread or distribute in Hawaiian, is an abstract framework for distributed sensor networks that provides a means for big data management as well as augmenting a DSN with the ability to adaptively optimize its bandwidth, detection, classification, and energy usage.

The Laha framework is made up of six layers that are stacked in a pyramid. Data entering the Laha framework enters into the bottom of the pyramid. As data moves upward through the layers, noise is discarded, less interested events are discarded or aggregated into upper layers, events are given more meaning and context, associations and predictions are made. 

\subsection{Big Data Management in Laha}
The Laha framework acts as an adaptive sieve for filtering noise and uninteresting data collected from a DSN. In this way, each layer only passes what it considers interesting to the layer above it. All data at a particular layer is garbage collected at specific intervals relating to its important to the DSN.

Each layer only keeps data for a specified amount of time before it is garbage collected. As data moves up the pyramid, it is generally considered more useful and therefore has a longer Time to Live (TTL), the amount of the time the data lives before it is garbage collected.  When a higher layer detects "something interesting", the data contained in the time window of "something interesting" is copied into the layers above it and will still persist even though the original data is garbage collected. In this way, we preserve data from all layers if they are associated with interesting data. 
% TODO: Figure of Laha layers here

\subsubsection{Instantaneous Measurements}
The Instantaneous Measurements Layer (IML) receives raw, sampled data from the DSN. The amount of data received is determined by the sample rate of each device multiplied by the number of fields per sample. Most of the time samples, from devices in the network are mainly sampling noise. A large percentage of the data in this layer is destined for garbage collection and data is assigned a Time to Live (TTL) of one hour. 


\subsubsection{Aggregate Measurements}

\subsubsection{Detections}

\subsubsection{Incidents}

\subsubsection{Phenomena}

\subsection{Phenomena: Providing Adaptive Optimizations in Laha}
