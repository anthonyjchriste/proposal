%%%%%%%%%%%%%%%%%%%%%%%%%%%%%% -*- Mode: Latex -*- %%%%%%%%%%%%%%%%%%%%%%%%%%%%
%% uhtest-abstract.tex -- 
%% Author          : Robert Brewer
%% Created On      : Fri Oct  2 16:30:18 1998
%% Last Modified By: Robert Brewer
%% Last Modified On: Fri Oct  2 16:30:25 1998
%% RCS: $Id: uhtest-abstract.tex,v 1.1 1998/10/06 02:06:30 rbrewer Exp $
%%%%%%%%%%%%%%%%%%%%%%%%%%%%%%%%%%%%%%%%%%%%%%%%%%%%%%%%%%%%%%%%%%%%%%%%%%%%%%%
%%   Copyright (C) 1998 Robert Brewer
%%%%%%%%%%%%%%%%%%%%%%%%%%%%%%%%%%%%%%%%%%%%%%%%%%%%%%%%%%%%%%%%%%%%%%%%%%%%%%%
%% 

\begin{abstract}
I propose an abstract distributed sensor network framework, Laha, that adaptively optimizes triggering, collection, detection, classification, energy usage, and bandwidth. This is accomplished using three steps. First, by creating a data hierarchy that designates the importance of data within a network. In this way, signals of interest are kept while sensor noise is reduced. Second, create the concept of distributed sensor network Phenomena, a layer above signal classification that provides predictive capabilities within sensor networks. Third, utilize the predictive models to increase triggering efficiency, optimize signal detection, and decrease network sensor bandwidth and energy usage.
\end{abstract}
