%%%%%%%%%%%%%%%%%%%%%%%%%%%%%% -*- Mode: Latex -*- %%%%%%%%%%%%%%%%%%%%%%%%%%%%
%% uhtest-abstract.tex -- 
%% Author          : Robert Brewer
%% Created On      : Fri Oct  2 16:30:18 1998
%% Last Modified By: Robert Brewer
%% Last Modified On: Fri Oct  2 16:30:25 1998
%% RCS: $Id: uhtest-abstract.tex,v 1.1 1998/10/06 02:06:30 rbrewer Exp $
%%%%%%%%%%%%%%%%%%%%%%%%%%%%%%%%%%%%%%%%%%%%%%%%%%%%%%%%%%%%%%%%%%%%%%%%%%%%%%%
%%   Copyright (C) 1998 Robert Brewer
%%%%%%%%%%%%%%%%%%%%%%%%%%%%%%%%%%%%%%%%%%%%%%%%%%%%%%%%%%%%%%%%%%%%%%%%%%%%%%%
%% 

%% Revision notes
%Abstract should include short summary of experimental design and proposed contributions. One should be able to read the abstract and get the gist of the entire proposal without reading anything else. 

%One of these words is not like the others (i.e. "energy usage"). All the others are domain-independent. 

%Can you rework this to split out the domain-dependent and domain-independent features?

\begin{abstract}
I propose an abstract distributed sensor network framework, Laha, that adaptively optimizes triggering, collection, detection, classification, energy usage, and bandwidth. This is accomplished in the following way. First, Laha provides a data hierarchy whose levels represent the importance of data within a network. Laha uses this data hierarchy to simultaneously retain signals of interest and filter out sensor noise. Second, the top level of the Laha's data hierarchy, called "Phenomena", goes beyond simple signal classification by providing predictive capabilities. Third, Laha can use information from the Phenomena level of the data hierarchy as a feedback mechanism to increase triggering efficiency, optimize signal detection, and decrease network sensor bandwidth and energy usage.

Laha's data hierarchy includes the following levels. At each level, we further refine the data which is forwarded to higher layers in the hierarchy and filter out sensor noise. The lowest level, the Instantaneous Measurements Layer (IML) receives raw, sampled data from the network and forwards aggregate measurements to the Aggregate Measurements Layer (AML). The AML consumes feature extracted data streams to provide triggers for possible signals of interest which are forwarded to the Detections Layer (DL). The DL receives raw data from a subset of network sensors corresponding to time windows as determined in the AML. The DL data is forwarded to the Incidents Layer (IL) where standard and state of the art algorithms are used to identify and classify signals of interest within the windows provided by the DL. Finally, data from the DL is forwarded to the Phenomenon Layer (PL) where context, causality, and predictive analytics are created. Phenomena are then used to adaptively tune and optimize the lower levels of the hierarchy.

The Laha framework aims to provide the following contributions. Laha design, a novel distributed sensor network adaptive design with seven useful benefits. An empirical evaluation that confirms or denies the seven useful properties. OPQMauka, a reference implementation of Laha designed as a middleware for a distributed power quality network. Lokahi, a reference implementation of Laha designed as a middleware for a distributed infrasound and sensor agnostic network. New implications for modern distributed sensor networks. 

Laha will be evaluated by designing and implementing two Laha-compliant reference implementations (OPQMauka and Lokahi). The reference implementations will be deployed to test sites. The data collected by Laha will be evaluated by comparing the data to industry standard sensors deployed within the same sensing field. I will use the Laha deployment to evaluate the proposed benefits of the framework.

I expect to deploy reference implementations before the end of 2018 with data collecting beginning and continuing through Q3 2019. I anticipate writing my dissertation along side the deployment and data collection process and to be finished in Q3 2019. 

% Proposed contributions
\end{abstract}
